\section{Theory}
This section presents the theoretical relationships that are necessary to understand the circuit under investigation. Cover in detail the actual circuit designs and the basis of predictions that you use.  \textbf{This section will leave no doubt concerning your understanding of the circuit’s theoretical performance.}

\begin{enumerate}
	\item Don’t just restate the pre-laboratory assignment.  Redevelop the theory supporting the equations.
	\item \underline{All} measurements you plan on taking will be predicted in this section!
\end{enumerate}

\subsection{LaTex Examples}

\begin{figure}[h]
	\centering
	\includegraphics[width=0.7\textwidth]{test}
	\caption{Fermi Distribution for an Intrinsic Semiconductor\cite{saleh2007fundamentals}}
	\label{fig:test1}
\end{figure}

One can reference figure \ref{fig:test1} in the document. The figure is on page \pageref{fig:test1}.

\bigskip
I can even add a footnote. \footnote{I am a footnote.}
Or add the following quote

\begin{quote}
	"For any academic/research writing, incorporating references into a document is an important task. Fortunately, LaTeX has a variety of features that make dealing with references much simpler, including built-in support for citing references. However, a much more powerful and flexible solution is achieved thanks to an auxiliary tool called BibTeX (which comes bundled as standard with LaTeX). Recently, BibTeX has been succeeded by BibLaTeX, a tool configurable within LaTeX syntax."\cite{bib_manage}
\end{quote}