\section{Results and Discussion}
This is the main section of the report.  In this section, actual circuit performance characteristics are compared with the expected circuit performance.
\begin{enumerate}
	\item You must make a \underline{quantitative} comparison of the actual and predicted performance characteristics.  Percent error is defined as:
	
		\qquad $ \Bigl|\dfrac{\text{measured value - expected value}}{\text{expected 	value}}\Bigr|*100\% $
	
	\item Present all of your expected and measured data with their percent errors in tabular form before you begin your discussion.  Do not present the Theory Section predictions using standard values.  Recalculate expected values using the measured parameter values (resistances, capacitances, Vγ , etc).
	\item The results portion should contain tables that summarize all measurements versus theoretical values.  Do not place these tables in an appendix unless they are excessively long.  (Supporting documents are appendices.)  Any observations or sketches addressed by the discussion are presented here and appropriately labeled.
	\item Your discussion ties together the theory and measurements sections and then relates them to the introduction.
	\begin{enumerate}
		\item Discuss all measurements.
		\item Organize discussion to be easily readable in relation to the purpose of the report and as stated in the introduction.  Discussing particular measurements or classes of measurements with their errors does this the best
		\item Attempt to correspond errors to the measured and calculated data.  While describing errors, quantify the magnitude of the error and its sign.  The framework used for identifying probable sources of error is to consider parametric errors, measurement errors, and modeling errors.
		\begin{enumerate}
			\item Parametric errors occur when using general or standard component and device values to calculate expected results instead of the actual measured component values. 
			\item Measurement error is the result of measuring voltage, current, resistance, frequency and time with instruments of limited accuracy or precision.  Describing the errors associated with making the measurement is very dependent on the procedures used to obtain the data.
			\item Modeling errors account for the many approximations and simplifications used to arrive at simple theoretical relationships, thus resulting in limited model accuracy.  More complex models, such as those used in PSPICE\textsuperscript{\textregistered} and AWB\textsuperscript{\textregistered}, can be used to compare with experimental results when analytical models are less accurate.
		\end{enumerate}
		\item Finally, relate the results and the size of errors to the objective of the lab, (stated in your introduction).
	\end{enumerate}
	\item Explain the theoretical basis for the observed results in this section of the report.  At a minimum, a qualitative explanation of results is required.  A quantitative discussion is even better.  Pertinent tables, figures, and plots will be organized, labeled, and referred to by table and figure numbers in the report.
\end{enumerate}


\section{Conclusion and Recommendations}
This section concludes your report.  It is a summary of the lab results and findings in concise form.  \textbf{NO NEW IDEAS ARE INTRODUCED IN THIS SECTION!}  Include your recommendations on how you’d modify measurement procedures or design methods.  \textbf{\underline{Don’t} simply state that the lab worked}, but instead state what you learned and verified in performing the laboratory exercise.